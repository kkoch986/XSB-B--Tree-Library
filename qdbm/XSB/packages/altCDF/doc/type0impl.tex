\section{The Type-0 Query Interface} \label{sec:type0query}

There are two main design goals behind the Type-0 query interface.
\begin{itemize}
\item to provide a Prolog interface to CDF based on  
the axioms in Chapter~\ref{sec:type0}, and the {\sc inh} proof system
derived from \refsec{sec:inheritance} along with
Proposition~\ref{prop:necesscondinh}.
\item to provide a highly efficient and scalable interface to CDF.
\end{itemize}

Indeed, the Type-0 interface has been used to support CDF instances
containing nearly a million extensional facts that require heavy
manipulation and access, and are used as back-ends to interactive
graphical systems.  As discussed below, this need for efficiency
affects certain aspects of the interface.

\subsection{Virtual Identifiers}
As discussed, Type-0 instances do not make contain facts of the form
\pred{necessCond/2}.  In the implementation of CDF, {\tt necessCond/2}
goals can be called, and their implementation obeys the first-argument
inheritance for {\tt necessCond}.  However, it is important to note
that {\bf the Type-0 interface does not use information in virtual
identifiers} as the following example shows.

\begin{example} \rm 
Consider the CDF instance containing only the fact
{\small 
{\tt 
\begin{tabbing}
foo\=foo\=foo\=foo\=foo\=foo\=foooo\=foooooooooooooooo\=\kill
\> necessCond\_ext(cid(a,test),vid(exists(rid(r,test),cid(b,test)))).
\end{tabbing} } } 
%
\noindent
by the semantics of Chapter~\ref{chap:type1}, this CDF instance
logically entails {\small {\tt
\begin{tabbing}
foo\=foo\=foo\=foo\=foo\=foo\=foooo\=foooooooooooooooo\=\kill
\> hasAttr(oid(a,test),rid(r.text),cid(b,text))$^{\cI}$
\end{tabbing} } } 
%
\noindent
However the Type-0 interface will answer ``no'' to the query 
{\small 
{\tt 
\begin{tabbing}
foo\=foo\=foo\=foo\=foo\=foo\=foooo\=foooooooooooooooo\=\kill
\> ?- hasAttr(oid(a,test),rid(r.text),cid(b,text)).
\end{tabbing} } } 
%
\end{example}

\subsection{Computing Irredundant Answers}

Consider the running sutures example of Chapter~\ref{sec:type0} to
which is added a fact
%
{\small 
{\tt 
\begin{tabbing}
foo\=foo\=foo\=foo\=foo\=foo\=foooo\=foooooooooooooooo\=\kill
\> hasAttr(oid(sutureU245H),rid(needleDesign),cid('Adson Dura')).
\end{tabbing} } } 
%
\noindent
Suppose the query {\tt ?-
hasAttr(oid(sutureU245H),rid(needleDesign),Y)} were asked.  Via an
{\sc inh} proof, rhe CDF instance would imply the answers 
%
{\small {\tt
\begin{tabbing}
foo\=foo\=foo\=foo\=foo\=foo\=foooo\=foooooooooooooooo\=\kill
\>  hasAttr(oid(sutureU245H),rid(needleDesign),cid('Adson Dura')) \\
\> hasAttr(oid(sutureU245H),rid(needleDesign),cid(needleDesignTypes)) \\
\>  hasAttr(oid(sutureU245H),rid(needleDesign),cid('CDF Classes))
\end{tabbing} } }
%
\noindent
The last two answers are, of course, redundant according to
Definition~\ref{def:redund}.  Omitting redundant answers is important
both for human comprehension of information in a CDF instance, and to
reduce excessive backtracking in applications.

% TLS coversAttr
Computation of irredundant answers is done in CDF by creating a {\em
preference relation}\ \ on the relations {\tt hasAttr/3}, {\tt
classHasAttr/3}, {\tt allAttr/3}, {\tt minAttr/3}, {\tt maxAttr/3} and
{\tt necessCond} using the techniques of \cite{CuSw02}.  The schematic
code for a query to {\tt hasAttr/3} in which the first argument is
known to be bound, and the second two free, is shown in
Figure~\ref{fig:preference}.  Basic information concerning {\tt
hasAttr/3} within a CDF instance is kept via the predicate {\tt
immed\_hasAttr/3} (and similarly for other CDF relations), and {\tt
hasAttr/3} uses {\tt immed\_hasAttr/3} to compute implications via
inheritance upon demand.  In the compilation of the code in
Figure~\ref{fig:preference}, well-founded negation is used to ensure
that only preferred answers are returned.  It is easy to see by
comparing the preference rules of Figure~\ref{fig:preference} with
Propositions~\ref{prop:inh1}-\ref{prop:inh2}, that the preference rule
ensures that answers are returned only if they are not implied by
other answers.  Similar approaches are used for other query modes and
CDF relations.  

\begin{figure}[htb] 
%-------------------------------------------
{\small {\sf  
\begin{tabbing}
foo\=foo\=foooooooooooooooooooooooooooooooooooooooo\=foo\=\kill
%
\> hasAttr(X,Y,Z):- \\
\> \> 	nonvar(X), \\
\> \> 	(var(Y) -$>$ hasAttr\_bff(X,Y,Z) ; hasAttr\_bbf(X,Y,Z)). \\
	   \\
\> :- table hasAttr\_bbf/3. \> \> :- table hasAttr\_bff/3.\\
\> hasAttr\_bbf(X,Y,Z):-  \> \> hasAttr\_bff(X,Y,Z):-  \\
\> \> 	isa(X,XSup), 	\> \> 	isa(X,XSup), \\
\> \> 	isa(Y,YSup), 	\> \>  	immed\_hasAttr(XSup,Y,Z). \\
\>  \> 	immed\_hasAttr(XSup,YSup,Z). \\
\\
\> prefer(hasAttr\_bbf(X,y,Z1),hasAttr\_bbf(X,Y,Z2)):-  
\> \> prefer(hasAttr\_bff(X,Y1,Z1),hasAttr\_bff(X,Y2,Z2)):-  \\
\> \> 	isa(Z1,Z2),\pnot(Z1 = Z2). 
\> \> 	isa(Y1,Y2),\pnot(Y1 = Y2), \\
\> \> \> \> 	isa(Z1,Z2),\pnot(Z1 = Z2).
%
\end{tabbing} } }
\caption{Schematic Representation for Selected Modes of {\tt
hasAttr/3} Implementation} 
\label{fig:preference}
\end{figure}

\subsection{Implementations of {\tt isa/2}} \label{sec:isaimpl}

In implementing {\tt isa/2} there are a number of tradeoffs to be made
between semantic power and efficiency.  We discuss some of them here
in order to motivate the design of the Type-0 API.

\subsubsection{To Table or Not to Table}  Tabling {\tt isa/2} (or the
predicates that underly it) may have several advantages.  First,
consider the goal {\tt ?- isa(cid('CDF Root',cdf),X)} that traverses
through the entire {\tt isa/2} hierarchy.  Is {\tt isa/2} is tabled,
{\tt X} will be instantiated once for each class in the hierarchy.  If
{\tt isa/2} is not tabled, {\tt X} will be instantiated for every path
in the hierarchy whose initial class is {\tt cid('CDF Root',cdf)}.
Since the number of paths in a directed graph can be exponential in
the number of nodes in the graph, a failure to table {\tt isa/2} can
potentially be disasterous.  Whether it is or not depends on the
structure of the inheritance hierarchy.  To the extent the hierarchy
is tree-like, tabling {\tt isa/2} will not be of benefit, as the
number of paths from any node in a tree is equal to the number of
nodes in the tree.  Indeed, in such a case, tabling {\tt isa/2} could
be a disadvantage, as large parts of the hierarchy may need to be
materialized in a table.  On the other hand, if there is much multiple
inheritance in the hierarchy, tabling {\tt isa/2} can vastly improve
performance.
%
A second consideration is whether intensional rules are used in a CDF
instance, and if so, the form of the rules.  If intensional rules
themselves call predicates in the Type-0 interface, there is a risk of
infinite loops if {\tt isa/2} isn't tabled.

As a result of these considerations, certain predicates underlying
{\tt isa/2} are tabled.  However, this tabling can be removed by
reconfiguring and recompiling CDF.  To do this, the file {\tt
cdf\_definitions.h} in {\tt \$XSBHOME/packages/altCDF} must be edited,
changing the line

{\tt DEFINE TABLED\_ISA 1}

\noindent to

{\tt DEFINE TABLED\_ISA 0}

\noindent
and recompiling {\tt cdf\_init\_cdf.P}.

\subsubsection{Product Classes}
%
From an operational perspective however, a query @tt{?- isa(X,Y)} can
easily be intractable if product classes are used.

\begin{example} \rm
Let {\tt cid(boolean,s)} be a class with two subclasses: {\tt
oid(true,s)} and {\tt oid(false,s)}.  Then the product class {\tt
cid(f(cid(boolean,s),...,cid(boolean,s),s)} will contain a number of
subclasses exponential in the arity of {\tt f}.
\end{example}

In order to use product classes in practical applications the
implementation of {\tt isa/2} distinguishes a general isa relation in
which a given fact may be proved using Instance Axioms, the Domain
Containment Axiom (Axiom~\ref{ax:contained}) and the Implicit Isa
Axiom (Axiom~\ref{ax:implsc}) from {\em explicit} isa proved without
the Implicit Subclassing Axiom.  Based on this distinction, two
restrictions are made:

\begin{enumerate}

\item {\em Restriction 1}: Axioms used to prove answers to the query
{\tt ?- isa(X,Y)} depend on the instantiation of {\tt X} and {\tt Y}.

\item {\em Restriction 2}: If {\tt immediate\_isa(Id1,Id2)} is true then
{\tt Id2} is an atomic identifier.
\end{enumerate}

We discuss each restriction in turn.  The behavior of {\tt isa/2} for
various instantiation patterns is as follows.

\begin{enumerate} 

\item If {\tt Id1} and {\tt Id2} are both ground, the Implicit
Subclassing Axiom is used, if necessary.

\item If {\tt Id1} is not ground, the Implicit Subclassing Axiom is
{\em not} used, in order to avoid returning a number of answers that
is exponential in the size of a product identifier.

\item If {\tt Id1} is ground but not {\tt Id2} then the Implicit
Subclassing Axiom may used in the first step of a derivation.  In
other words, in any isa derivation for this instantiation pattern, the
first step may use the Implicit Subclassing Axiom to "match" a term in
the {\tt immediate\_isa/2} relation, but subsequent steps must use
explicit isa.  Upon backtracking the Implicit Subclassing Axiom may be
used again to begin a new derivation, but subsequent steps in this
derivation must cannot use this axiom.
\end{enumerate}

The second assumption helps to reinforce the assumption of case 3
above.  Without it, users might expect that the Implicit Subclassing
Axiom could be used in each step of a derivation of an {\tt isa/2}
fact.  Such an implementation would slow down the execution of {\tt
isa/2} so that it would be unusable for many
applications~\footnote{Given Restriction 2, atomic identifiers usually
occur within the inner loops of {\tt isa/2}.  Atomic identifiers have
the advantage that these inner loops can use unification to traverse
the hierarchy.  If product identifers are used, they must be
abstracted using {\tt functor/3} and the hierarchies of their inner
arguments traversed, a much slower method.}.

\begin{example}  \rm Suppose we have the following CDF instance.

{\small {\tt
\begin{tabbing}
foo\=foo\=foooooooooooooooooooooooooooooooooooooooo\=foo\=\kill
%
\> isa\_ext(cid(bot,s),cid(mid,s)). \\
\> isa\_ext(cid(mid,s),cid(top,s)). \\
\\
\> isa\_ext(cid(prod(cid(mid,s),cid(top,s)),s),cid(myClass,s)).
\end{tabbing} } }

\begin{itemize}
%
\item The query 
{\small {\tt
\begin{tabbing}
foo\=foo\=foooooooooooooooooooooooooooooooooooooooo\=foo\=\kill

\> ?- isa(cid(prod(cid(bot,s),cid(mid,s)),s),cid(prod(cid(mid,s),cid(mid,s)),s)
\end{tabbing} } }

would succeed.

\item 
{\small {\tt
\begin{tabbing}
foo\=foo\=foooooooooooooooooooooooooooooooooooooooo\=foo\=\kill

\> ?- isa(cid(prod(cid(bot,s),cid(mid,s)),s),X)
\end{tabbing} } }
would successively unify {\tt X} with 
{\small {\tt
\begin{tabbing}
foo\=foo\=foooooooooooooooooooooooooooooooooooooooo\=foo\=\kill
\> (1) cid(prod(cid(bot,s),cid(mid,s)),s), \\
\> (2) cid(prod(cid(mid,s),cid(top,s)),s), \\
\> (3) cid(myClass,s), {\rm and} \\
\> (4) cid('CDF Root',cdf)
\end{tabbing} } }

\item The query
{\small {\tt
\begin{tabbing}
foo\=foo\=foooooooooooooooooooooooooooooooooooooooo\=foo\=\kill

\> ?- isa(X,cid(prod(cid(mid,s),cid(mid,s)),s))
\end{tabbing} } }
would unify {\tt X} only with {\tt
cid(prod(cid(mid,s),cid(mid,s)),s)}

\end{itemize}
\end{example}

\subsection{The Type-0 API} \label{sec:type0api}

Exceptions to all predicates in this API are based on the context {\tt
query} (See \refsec{sec:consist}).

\begin{description}

%----------------------------------------------------------------
\mycomment{
\ourpreditem{implicit\_isa/2}  {\tt implicit\_isa(Id1,Id2)} forms a partial
implementation of the Implicit Subclassing Axioms for product
identifiers\ref{???}.  As an example of implicit isaing of product
classes, {\tt id(f(id(a,source1),id(b,source2),source3)} is a subclass
of {\tt id(f(id(c,source1),id(b,source2),source3)} if {\tt
id(a,source1)} is a subset of {\tt id(a,source1)}.  Because the use of
product identifiers can isa relations that are exponential in the size
of the product identifiers, the implementation described below
attempts to partially traverse the implicit isa relation in a manner
that is semantically meaningful while also remaining tractable.

The semantics of {\tt implicit\_isa/2} is mode-dependent.  Let fully
ground inputs be treated as {\tt +} and non-fully ground inputs treated
as {\tt -}.  Suppose we have a call {\tt implicit\_isa(C1,C2)}:

\begin{itemize} 
\item {\tt implicit\_isa(+,+)}: succeeds if {\tt C1} is
not equal to {\tt C2} and {\tt C1} is lower than {\tt C2} on the isa
hierarchy by the isa axioms.

\item {\tt implicit\_isa(+,-)}: succeeds if {\tt C1} \=
{\tt C2}, {\tt C1} is a subclass, (member, etc) of {\tt C2} by the isa
axioms {\em and} for some {\tt C3} {\tt immed\_isa(C2,C3)} is true.

\item {\tt implicit\_isa(-,+)}: fails.

\item {\tt implicit\_isa(-,-)}: fails.
\end{itemize}

The motivation for this partial implementation is as follows.  If both
terms are ground, determining their relation in the isa hierarchy is
linear in the sizes of the terms.  In all cases where variables are
present, there is the possibility of backtracking through a large
isa\_relation.  For the instantiation pattern {\tt immed\_isa(+,-)}
this is addressed by searching through only those product identifiers
that occur in the first argument of the immediate isa relation.
Because of the assumption that product identifiers can occur only in
the first argument of the immediate isa relation, this option is not
available for the instantiation patterns {\tt implicit\_isa(-,+)} and
{\tt implicit\_isa(-,-)}, so they fail. }
%----------------------------------------------------------------

\ourpredmoditem{isa/2}{cdf\_init\_cdf} The operational semantics of
{\tt isa/2} is defined in \refsec{sec:isaimpl}.

\mycomment{
/* TLS: the supporting predicates for isa/2 may or may not be tabled.
Certain of the CDF operations depend on the prolog semantics of isa.
Rather than changing these predicates, I moved isa tabling to a lower
level, past mode checks, and the first call to isa in each mode.  This
should cause no extra tabling beyond tabling isa/2, and perhaps a bit
less tabling.  If you definately want tabled behavior use
table\_isa/2.  Note that explosive\_isa/2, proper\_isa/2*/}

\ourpredmoditem{explosive\_isa/2}{cdf\_init\_cdf}
{\tt explosive\_isa(Sub,Sup)} follows the isa axioms for product
identifiers rather than the algorithm of {\tt isa/2}. Thus if neither
{\tt Id1} nor {\tt Id2} are product identifiers, or if {\tt Id1} and
{\tt Id2} are fully ground product identifiers, {\tt explosive\_isa/2}
behaves as {\tt isa/2}.  Otherwise, suppose {\tt Id1} is a (perhaps
partially ground) product identifier whose Nid has the outer functor
{\tt F/A}.  If the Nid of {\tt Id2} is a variable, it is instantiated
to a skeleton of {\tt F/N}; otherwise its outer functor must be {\tt
F/A}.  In either case, both Nids are broken into their constituent
identifiers and {\tt explosive\_isa/2} is recursively called on each
of these.  {\tt explosive\_isa/2} thus removes Restriction 1 above,
but not Restriction 2.

\ourpredmodrptitem{allAttr/3}{cdf\_init\_cdf}

\ourpredmodrptitem{hasAttr/3}{cdf\_init\_cdf}

\ourpredmodrptitem{maxAttr/4}{cdf\_init\_cdf}

\ourpredmodrptitem{minAttr/4}{cdf\_init\_cdf}

\ourpredmoditem{classHasAttr/3}{cdf\_init\_cdf}
%
These predicates assume they are operating on a CDF instance, $\cO$ in
which the {\tt isa/2} relation is acyclic.  For efficiency reasons,
given a goal, $G$, the behavior of these predicates further depends on
whether various arguments of $G$ are ground atomic identifiers. 
\begin{itemize}
\item  If either the first argument of $G$ is a ground atomic
identifier, or the second and third arguments of $G$ are ground atomic
identifiers, each answer $G\theta$ will be a member of a set $S$ which
is the most specific irredundant set containing only elements that
unify with $G$.
%
\item Otherwise, each answer $G\theta$ will be a member of a set $S$ that 
is the most specific irredundant set containing only elements that
unify with $G$.
\end{itemize}


\ourpredmoditem{nessesCond/2}{cdf\_init\_cdf}
Given a goal {\tt nessesCond(?Id,-Vid)}, each answer
$nessesCond(Id,Vid)\theta$ will be a member of a set $S$ which is the
most specific irredundant set containing only elements that unify with
{\tt nessesCond(?Id,-Vid)}.  

\ourpreddomitem{isType0Term/1}{cdf\_checks}
{\tt isType0Term(?Term)} succeeds if {\tt Term} unifies with an
extensional fact (e.g. a term of the form {\tt isa\_ext(A,B)}, {\tt
hasAttr\_ext(A,B)}, etc.), an intensional rule head (e.g. a term of
the form {\tt isa\_int(A,B)}, {\tt hasAttr\_int(A,B)}, etc.), or a
semantic type-0 predicate (e.g. a term of the form {\tt isa(A,B)},
{\tt hasAttr(A,B)}, etc.).

\end{description}

%TLS: check out loops in hasAttr.
