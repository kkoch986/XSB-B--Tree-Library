
\chapter{{\tt B+ Tree}: B+ Tree Indexing Library}

\begin{center}
  {\Large {\bf By Ken Koch}}
\end{center}

\section{Introduction}

B+ Trees are data structures which provide fast indexing for range and prefix queries. 
The XSB B+ Tree library uses an external library to implement an on disk B+ Tree indexing system.
Currently, the B+ Tree library uses a library called Kyoto Cabinet;
%% 
\begin{quote}
  \url{http://fallabs.com/kyotocabinet/} 
\end{quote}
%% 
This library supports a lot of the features provided by the kyoto library including transaction support.
To use the XSB B+ Tree library, you must first download and install the Kyoto Cabinet Library.

\section{Basic Usage}

The basic operations on a B+ Tree can be carried out using a small set of predicates for create, open, insert and get.

\subsection{Creating and Opening a B+ Tree}

Most B+ Tree indexes will be created explicitly using {\tt btcreate/3} and {\tt btinit/2}. 
These two predicates provide the functionality to create a B+ Tree in the working directory and 
get a handle on a tree that you have created to allow one to work with it. To begin, you must consult the B+ 
tree library:
%%
\begin{verbatim}
 ?- [bt].  
\end{verbatim}
%%
This will load all of the predicates we need. Next to create a B+ Tree you will use {\tt btcreate/3}:

\begin{description}

  \index{\texttt{btcreate/3}}
\item[btcreate({\it +DBName}, {\it +PredicateName/Arity}, {\it +IndexArgumentNumber})]\mbox{}
  \\
    {\it DBName} is an atom that specifies a filename where the B+ Tree will be placed (relative to the working directory).

    {\it PredicateName} is an atom that specifies the predicate that will be stored in this B+ Tree.
    
    {\it Arity} is a number which corresponds to the arity of the predicate which will be stored in the B+ Tree.
    
    {\it IndexArgumentNumber} is a number which corresponds to the (1-based) argument of the predicate to build the 
    B+ Tree index on, currently the library does not support multiple indexes. 
  \index{\texttt{btinit/2}}
\item[btinit({\it +DBName}, {\it -Handle})]\mbox{}
  \\
    {\it DBName} is an atom that specifies a filename where the B+ Tree is placed (relative to the working directory).

    {\it Handle} will be unified with an integer which needs to be passed to future library calls.

\end{description}
One should always close handles that are opened using {\tt binit/2}, this ensures proper termination of the tree
and that changes made are actually comitted to disk.
\begin{description}

  \index{\texttt{btclose/1}}
\item[btclose({\it +Handle})]\mbox{}
  \\
    {\it Handle} is an integer returned by {\tt btinit/2}

\end{description}

\subsection{Inserting Data}

To insert data into the B+ Tree, you should use the {\tt btinsert/2} predicate which takes as arguments the handle to
your B+ Tree and the value you would like to insert. Values are inserted as functor symbols and the libarary will extract
the correct value to use as the index.

\begin{description}

  \index{\texttt{btinsert/2}}
\item[btinsert({\it +Handle}, {\it +Value})]\mbox{}
  \\
    {\it Handle} is an integer returned by {\tt btinit/2}

    {\it Value} is a functor symbol with the prediacte name and arity used when creating the index.

\end{description}

For example to create a B+ Tree index for the predicate p/2 indexed on the first argument and insert some values into it:
%% 
\begin{verbatim}
  ?- [bt].
  ?- btcreate(myindex, p/2, 1). 
  ?- btinit(myindex, Handle), 
      btinsert(Handle, p(a,b)), 
      btinsert(Handle, p(b,c)), 
      btclose(Handle).
\end{verbatim}
%% 
\subsection{Basic Querying}

To retrieve data from the index, we can use the basic query operation, {\tt btget/3}.

\begin{description}

  \index{\texttt{btget/3}}
\item[btget({\it +Handle}, {\it +Key}, {\it -Value})]\mbox{}
  \\
    {\it Handle} is an integer returned by {\tt btinit/2}

    {\it Key} The key to look up.

    {\it Value} is a functor symbol with the predicate name and arity used when creating the index.

\end{description}

You can then fail through the results in the B+ Tree that match the provided key.

The library also provides a predicate to retrieve all entries in the database regardless of key.

\begin{description}

    \index{\texttt{btgetall\_rev/2}}
  \item[btgetall({\it +Handle}, {\it -Value})]\mbox{}
  \\
    {\it Handle} is an integer returned by {\tt btinit/2}

    {\it Value} is a functor symbol with the predicate name and arity used when creating the index.

   \index{\texttt{btgetall\_rev/2}}
  \item[btgetall\_rev({\it +Handle}, {\it -Value})]\mbox{}
  \\
    {\it Handle} is an integer returned by {\tt btinit/2}

    {\it Value} is a functor symbol with the predicate name and arity used when creating the index.

\end{description}

{\tt btgetall\_rev/2} will provide the same output as {\tt btgetall/2} except it will return the records in reverse order.

\subsection{Manual Cursor Mode (MCM)}

The B+ Tree also provides functionality to manually control the database cursor and retrieve keys and values at any location in the tree.
To use MCM you must first intialize MCM on an open handle.

\begin{description}

  \index{\texttt{btminit/1}}
\item[btminit({\it +Handle})]\mbox{}
  \\
    {\it Handle} is an integer returned by {\tt btinit/2}, this will set the mode of this handle to manual cursor mode.

\end{description}

Next you can use MCM to move the cursor around the tree.

\begin{description}

  \index{\texttt{btmfirst/1}}
\item[btmfirst({\it +Handle})]\mbox{}
  \\
    {\it Handle} is an integer returned by {\tt btinit/2}, this will set cursor to the very first record in the tree.
  \index{\texttt{btmlast/1}}
\item[btmlast({\it +Handle})]\mbox{}
  \\
    {\it Handle} is an integer returned by {\tt btinit/2}, this will set cursor to the very last record in the tree.
  \index{\texttt{btmnext/1}}
\item[btmnext({\it +Handle})]\mbox{}
  \\
    {\it Handle} is an integer returned by {\tt btinit/2}, this will set cursor to the very next record in the tree.
  \index{\texttt{btmprev/1}}
\item[btmprev({\it +Handle})]\mbox{}
  \\
    {\it Handle} is an integer returned by {\tt btinit/2}, this will set cursor to the record immediately before the current record in the tree.

\end{description}

The following two predicates can be used to see the key and value at the current cursor position.

\begin{description}

  \index{\texttt{btmkey/2}}
\item[btmkey({\it +Handle}, {\it -Key})]\mbox{}
  \\
    {\it Handle} is an integer returned by {\tt btinit/2}
    {\it Key} is an atom containing the key of record at the current position.

  \index{\texttt{btmval/2}}
\item[btmval({\it +Handle}, {\it -Val})]\mbox{}
  \\
    {\it Handle} is an integer returned by {\tt btinit/2}
    {\it Val} is an atom containing the value of record at the current position.

\end{description}

An example of manual cursor mode can be found in the implementation of btgetall/2:

\begin{verbatim}
  btgetall(Handle, Value) :- 
    btminit(Handle),
    btmfirst(Handle),
    btgetallnext(Handle, Value).

  btgetallnext(Handle, Value) :- 
    btmval(TreeHandle, Value0),
    ( 
      unnumbervars(Value0, 0, Value1),
      Value = Value1 ; 
        btmnext(TreeHandle),
        btgetallnext(TreeHandle, Value)
    ).
\end{verbatim}

\subsection{Transactions}

The Library has basic support for transactions through the predicates {\tt btstrans/1}, {\tt btctrans/1}, and {\tt btatrans/1}.

\begin{description}

  \index{\texttt{btstrans/1}}
\item[btstrans({\it +Handle})]\mbox{}
  \\
    {\it Handle} is an integer returned by {\tt btinit/2}, this call will start a transaction on the given database handle.
  \index{\texttt{btctrans/1}}
\item[btctrans({\it +Handle})]\mbox{}
  \\
    {\it Handle} is an integer returned by {\tt btinit/2}, this call will commit a transaction on the given database handle.
  \index{\texttt{btatrans/1}}
\item[btatrans({\it +Handle})]\mbox{}
  \\
    {\it Handle} is an integer returned by {\tt btinit/2}, this call will abort a transaction on the given database handle.

\end{description}

\subsection{Meta Data}

The library also provides some functions for retrieving meta data about open databases.

\begin{description}

  \index{\texttt{btpredname/2}}
\item[btpredname({\it +Handle}, {\it -PredName})]\mbox{}
  \\
    {\it Handle} is an integer returned by {\tt btinit/2}.

    {\it PredName} will be an atom representing the prediacte name for this index.
  \index{\texttt{btarity/2}}
\item[btarity({\it +Handle}, {\it -Arity})]\mbox{}
  \\
    {\it Handle} is an integer returned by {\tt btinit/2}.

    {\it Arity} will be a number representing the arity used for this index
  \index{\texttt{btindexon/2}}
\item[btindexon({\it +Handle}, {\it -Argnum})]\mbox{}
  \\
    {\it Handle} is an integer returned by {\tt btinit/2}.
    
    {\it Argnum} will be a number representing the argument index for this index.

\end{description}

\subsection{Built-in Indexing}

The current implementation supports a built-in form of indexing as well with some temporary restrictions. 
The current implementation supports a read-only method to access facts stored in a database using XSB's {\tt import} construct.
To load an index using import, you must place an import statement inside a .P file like so:

\begin{verbatim}
  :- import q/2 from myindex as btree.
\end{verbatim}

Note that the predicate you load does not have to match the predicate name stored in the index but it does have to match 
on arity. Note following the {\tt as} keyword must be the term btree or btree(N) where N is the argument number to index on. Be 
aware this must also match the index argument used to create the index file.

Once the B+ Tree is loaded this way, you can query it by simply providing the predicate like so:

\begin{verbatim}
  ?- q(A,B).
\end{verbatim}

This will allow you to fail through all of the facts in the index. If the indexed argument is bound, a keyed lookup will be done, if it is not,
then manual cursor mode will be usede to retrieve the entries.

